
# informe.tex

```latex
\documentclass[12pt, a4paper]{article}
\usepackage[utf8]{inputenc}
\usepackage[spanish]{babel}
\usepackage{graphicx}
\usepackage{hyperref}
\usepackage{float}
\usepackage{booktabs}
\usepackage{longtable}

\title{Informe de Práctica: Estructura y Función del Genoma}
\author{Doctorando en Biología Computacional \\ Universidad San Sebastián}
\date{\today}

\begin{document}

\maketitle

\section{Introducción}
Este informe documenta el análisis bioinformático del gen \textit{NOS3} y elementos relacionados como parte de la práctica de laboratorio del curso troncal 2025 del Doctorado en Biología Computacional.

\section{Metodología}
Se utilizaron herramientas bioinformáticas y bases de datos públicas mediante:

\begin{itemize}
\item Consultas a NCBI via Entrez API
\item Alineamientos de secuencias con Biopython
\item Búsqueda de dominios proteicos
\item Análisis de contexto genómico
\end{itemize}

\section{Resultados}

\subsection{Análisis del gen NOS3}

\subsubsection{Ubicación genómica}
El gen \textit{NOS3} se encuentra ubicado en...

\begin{figure}[H]
\centering
\includegraphics[width=0.8\textwidth]{images/nos3_genomic_location.png}
\caption{Ubicación genómica del gen NOS3}
\label{fig:nos3_loc}
\end{figure}

\subsubsection{Secuencias obtenidas}
Se obtuvieron las siguientes secuencias:

\begin{longtable}{lll}
\toprule
Tipo & Accesión & Longitud \\
\midrule
Genómica & NC\_000007.14 & 21,026 bp \\
mRNA & NM\_000603.4 & 4,052 bp \\
\bottomrule
\end{longtable}

\subsubsection{Alineamiento}
El alineamiento entre las secuencias genómicas y de mRNA mostró:

\begin{verbatim}
[Resultados del alineamiento aquí]
\end{verbatim}

\subsection{Variantes de splicing}
Se identificaron X variantes de splicing:

\begin{itemize}
\item Variante 1:...
\item Variante 2:...
\end{itemize}

\subsection{Expresión tisular}
El gen \textit{NOS3} se expresa principalmente en:

\begin{itemize}
\item Tejido endotelial
\item Células musculares lisas
\end{itemize}

\section{Discusión}

Los resultados obtenidos demuestran que... [Análisis crítico de los hallazgos]

\section{Conclusiones}

\begin{itemize}
\item El gen \textit{NOS3} presenta...
\item Se confirmó la relación con \textit{ATG9B}...
\item Las proteínas derivadas de \textit{CALCA}...
\end{itemize}

\section*{Anexos}

\subsection*{Código Fuente}
El código utilizado está disponible en: \url{https://github.com/tuusuario/practica-genoma}

\subsection*{Archivos Generados}
Todos los archivos resultantes se encuentran en la carpeta \texttt{resultados\_lab/}

\end{document}
